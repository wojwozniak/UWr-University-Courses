\documentclass{article}
\usepackage[utf8]{inputenc}

\begin{document}

\title{Katastrofa Ariane 5 z 1996r}
\author{}
\date{}

\maketitle

\section*{Wprowadzenie}
Problem podobny do wspomnianego w pierwszej części zadania czwartego zdarzył się w 1996 roku w Gujanie Francuskiej. Zaledwie 40 sekund po starcie wybuchła rakieta Ariane 5 Europejskiej Agencji Kosmicznej. Prace nad rakietą trwały ponad dekadę, straty przekroczyły 500 milionów dolarów --- a wszystko przez jedną linię kodu i problem z konwersją liczby zmiennopozycyjnej na całkowitą.


\section*{Geneza problemu}
Kluczowym aspektem tej katastrofy były błędy związane z konwersją liczb. Oprogramowanie Ariane 5 używało 64-bitowych liczb zmiennopozycyjnej do reprezentacji prędkości horyzontalnej rakiety. W jednym miejscu, liczby te były konwertowane na 16-bitową liczbę całkowitą. Otrzymana liczba była większa niż 32767 - a więc większa niż mogła się w tej zmiennej zmieścić.

\section*{Przebieg awarii}
Powyższy problem spowodował zajęcie części pamięci zajmowanej wcześniej przez system sterowania silnikami rakiety. Mimo uruchomienia systemu awaryjnego, nie udało się utrzymać poprawnego działania programu. Silniki zaczęły pracować z maksymalną mocą, co doprowadziło do wybuchu.


Winna awarii była poniższa linia kodu (język Ada):
\begin{verbatim}
P_M_DERIVE(T_ALG.E_BH) := 
  UC_16S_EN_16NS (TDB.T_ENTIER_16S ((1.0/C_M_LSB_BH) * G_M_INFO_DERIVE(T_ALG.E_BH)));
\end{verbatim}

\section*{Podsumowanie}

Jedna linia kodu spowodowała ponad pół miliarda dolarów (a uwzględniając inflację prawie miliard) strat i znacząco spowolniła rozwój europejskiego programu kosmicznego. Tak samo jak historia o awarii Patriota z 1991 roku pokazuje ona, jak ważne jest dokładne pisanie kodu (i ostrożne podchodzenie do arytmetyki zmiennopozycyjnej).

\end{document}
 